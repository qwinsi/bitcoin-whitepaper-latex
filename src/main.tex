% Preamble
\documentclass{article}

% Packages
\usepackage[margin=1.5in]{geometry}
\usepackage{amsmath}
\usepackage{enumitem}
\usepackage{graphicx}
\usepackage{tikz}
\usepackage{pgffor} % \foreach

% https://tex.stackexchange.com/questions/28111/how-to-place-abstract-text-in-front-of-abstract-key
\renewenvironment{abstract}{\noindent\bfseries\abstractname:\normalfont}{}

% Document
\title{Bitcoin: A Peer-to-Peer Electronic Cash System}
\author{Satoshi Nakamoto \\ satoshi@gmx.com \\ www.bitcoin.org}
\date{} % don't show date

\begin{document}
\maketitle
\begin{abstract}
    A purely peer-to-peer version of electronic cash would allow online payments to be sent directly from one party to
    another without going through a financial institution.
    Digital signatures provide part of the solution,
    but the main benefits are lost if a trusted third party is still required to prevent double-spending.
    We propose a solution to the double-spending problem using a peer-to-peer network.
    The network timestamps transactions by hashing them into an ongoing chain of hash-based proof-of-work,
    forming a record that cannot be changed without redoing the proof-of-work.
    The longest chain not only serves as proof of the sequence of events witnessed,
    but proof that it came from the largest pool of CPU power.
    As long as a majority of CPU power is controlled by nodes that are not cooperating to attack the network,
    they'll generate the longest chain and outpace attackers.
    The network itself requires minimal structure.
    Messages are broadcast on a best effort basis,
    and nodes can leave and rejoin the network at will,
    accepting the longest proof-of-work chain as proof of what happened while they were gone.
\end{abstract}

\section{Introduction}\label{sec:introduction}
Commerce on the Internet has come to rely almost exclusively on financial institutions serving as
trusted third parties to process electronic payments.
While the system works well enough for most transactions, it still suffers from the inherent weaknesses of the trust based model.
Completely non-reversible transactions are not really possible, since financial institutions cannot
avoid mediating disputes.
The cost of mediation increases transaction costs, limiting the
minimum practical transaction size and cutting off the possibility for small casual transactions,
and there is a broader cost in the loss of ability to make non-reversible payments for non-reversible services.
With the possibility of reversal, the need for trust spreads.
Merchants must be wary of their customers, hassling them for more information than they would otherwise need.
A certain percentage of fraud is accepted as unavoidable.
These costs and payment uncertainties can be avoided in person by using physical currency, but no mechanism exists to make payments
over a communications channel without a trusted party.

What is needed is an electronic payment system based on cryptographic proof instead of trust,
allowing any two willing parties to transact directly with each other without the need for a trusted
third party.
Transactions that are computationally impractical to reverse would protect sellers
from fraud, and routine escrow mechanisms could easily be implemented to protect buyers.
In this paper, we propose a solution to the double-spending problem using a peer-to-peer distributed
timestamp server to generate computational proof of the chronological order of transactions.
The system is secure as long as honest nodes collectively control more CPU power than any
cooperating group of attacker nodes.

\section{Transactions}\label{sec:transactions}
We define an electronic coin as a chain of digital signatures.
Each owner transfers the coin to the next by digitally signing a hash of the previous transaction and the public key of the next owner
and adding these to the end of the coin.
A payee can verify the signatures to verify the chain of ownership.

\usetikzlibrary{
    arrows.meta,% [>=Triangle]
    fit,        % [fit=...]
    positioning % [right=of ...]
}

\tikzset{box/.style={draw, minimum size=2em, text width=3em, text centered},
    container/.style={draw, inner sep=20pt}
}

\begin{tikzpicture}[>=Triangle]
\foreach[evaluate={\prev=int(\i - 1)}] \i in {1, 2, 3} {
    \node (P\i-PubKey) at (\i * 5, 0) [box, text width=5em] {Owner \i's \\ Public Key};
    \node (P\i-Hash) [below=0.8cm of P\i-PubKey][box] {Hash};
    \node (P\i-Sig) [below=of P\i-Hash][box, text width=5em] {Owner \prev's \\ Signature};
    \draw [->] (P\i-Hash) -- (P\i-Sig);
    \draw [->] (P\i-Hash.45 |- P\i-PubKey.south) -- (P\i-Hash.45);
    \node (P\i-Tx) [fit=(P\i-PubKey)(P\i-Hash)(P\i-Sig)][container, label={[shift={(15ex,-4ex)}]north west:Transaction}]{};
    \node (P\i-PrivKey) [below=of P\i-Tx] [box, text width=6em] {Owner \i's \\ Private Key};
}

\draw [->] ([xshift=-2.5cm, yshift=0.5cm]P1-Hash.north) -| ([xshift=-0.3cm]P1-Hash.north);

\foreach[evaluate={\next=int(\i + 1)}] \i in {1, 2} {
    \draw [->, dashed] (P\i-PubKey.330) -- (P\i-PubKey.330 |- P\i-Hash.south east) -- node[sloped]{Verify} (P\next-Sig.165);
    \draw [->, dashed] (P\i-PrivKey.15) -- node[sloped]{Sign} (P\next-Sig.195);
    \draw [->] ([yshift=0.5cm]P\i-Hash.north -| P\i-Tx.east) -| ([xshift=-0.3cm]P\next-Hash.north);
}
\end{tikzpicture}

The problem of course is the payee can't verify that one of the owners did not double-spend the coin.
A common solution is to introduce a trusted central authority, or mint, that checks every
transaction for double spending.
After each transaction, the coin must be returned to the mint to
issue a new coin, and only coins issued directly from the mint are trusted not to be double-spent.
The problem with this solution is that the fate of the entire money system depends on the
company running the mint, with every transaction having to go through them, just like a bank.

We need a way for the payee to know that the previous owners did not sign any earlier transactions.
For our purposes, the earliest transaction is the one that counts, so we don't care
about later attempts to double-spend.
The only way to confirm the absence of a transaction is to be aware of all transactions.
In the mint based model, the mint was aware of all transactions and
decided which arrived first.
To accomplish this without a trusted party, transactions must be
publicly announced [1], and we need a system for participants to agree on a single history of the
order in which they were received.
The payee needs proof that at the time of each transaction, the
majority of nodes agreed it was the first received.

\section{Timestamp Server}\label{sec:timestamp-server}
The solution we propose begins with a timestamp server.
A timestamp server works by taking a
hash of a block of items to be timestamped and widely publishing the hash, such as in a
newspaper or Usenet post [2-5].
The timestamp proves that the data must have existed at the
time, obviously, in order to get into the hash.
Each timestamp includes the previous timestamp in
its hash, forming a chain, with each additional timestamp reinforcing the ones before it.

\usetikzlibrary{
    arrows.meta,% [>=Triangle]
    fit,        % [fit=...]
    positioning % [right=of ...]
}

\begin{tikzpicture}[>=Triangle]
\tikzset{box/.style={draw, minimum size=2em, text width=2em, text centered},
    container/.style={draw, inner sep=20pt}
}

\node (Hash) [box] at (-2, 1.5) {Hash};
\node (Item1) [box] at (-3, 0){Item};
\node (Item2) [box] [right=0.3cm of Item1] {Item};
\node (Item3) [box] [right=0.3cm of Item2] {...};
\node (Container) [container] [label={[shift={(8ex,-4ex)}]north west:Block}, fit=(Item1)(Item2)(Item3)] {};

\draw [->] (Container.north west)+(1, 0) |- ([yshift=-4]Hash.west);

\node (HHash) [box] at (4, 1.5) {Hash};
\node (IItem1) [box] at(3, 0) {Item};
\node (IItem2) [box] [right=0.3cm of IItem1] {Item};
\node (IItem3) [box] [right=0.3cm of IItem2] {...};
\node (CContainer) [container] [label={[shift={(8ex,-4ex)}]north west:Block}, fit=(IItem1)(IItem2)(IItem3)] {};

\draw [->] (CContainer.north west)+(1, 0) |- ([yshift=-4]HHash.west);

\draw [<-] ([yshift=4]Hash.west) --+ (-3, 0);
\draw [->] ([yshift=4]Hash.east) -- ([yshift=4]HHash.west);
\draw [->] ([yshift=4]HHash.east) --+ (3, 0);
\end{tikzpicture}


\section{Proof-of-Work}\label{sec:proof-of-work}
To implement a distributed timestamp server on a peer-to-peer basis, we will need to use a proof-
of-work system similar to Adam Back's Hashcash [6], rather than newspaper or Usenet posts.
The proof-of-work involves scanning for a value that when hashed, such as with SHA-256, the
hash begins with a number of zero bits.
The average work required is exponential in the number
of zero bits required and can be verified by executing a single hash.
For our timestamp network, we implement the proof-of-work by incrementing a nonce in the
block until a value is found that gives the block's hash the required zero bits.
Once the CPU
effort has been expended to make it satisfy the proof-of-work, the block cannot be changed
without redoing the work.
As later blocks are chained after it, the work to change the block
would include redoing all the blocks after it.

\usetikzlibrary{
    arrows.meta,% [>=Triangle]
    fit,        % [fit=...]
    positioning % [right=of ...]
}

\begin{tikzpicture}[>=Triangle]

\tikzset{box/.style={draw, minimum size=2em, text centered},
    container/.style={inner sep=20pt}
}

\foreach \i / \x in {0/-3, 1/3} {
    \node (P\i-PrevHash) [box] at (\x, 1) {PrevHash};
    \node (Nonce) [box] [right=0.3cm of P\i-PrevHash] {Nonce};
    \node (Tx1) [box] [below=0.3cm of P\i-PrevHash]{Tx};
    \node (Tx2) [box] [right=0.3cm of Tx1] {Tx};
    \node (Tx3) [box] [right=0.3cm of Tx2] {...};
    \node (P\i-Container) [container] [label={[shift={(8ex,-4ex)}]north west:Block}, fit=(P\i-PrevHash)(Tx1)(Tx2)(Tx3)] {};
    \draw (P\i-Container.north west) -- (P\i-Container.south west) -- (P\i-Container.south east) -- (P\i-Container.north east);
}

\draw [<-] (P0-PrevHash.west) --+ (-2, 0);
\draw [->] (P0-Container.east |- P1-PrevHash.west) -- (P1-PrevHash.west);
\end{tikzpicture}

The proof-of-work also solves the problem of determining representation in majority decision making.
If the majority were based on one-IP-address-one-vote, it could be subverted by anyone
able to allocate many IPs. Proof-of-work is essentially one-CPU-one-vote.
The majority
decision is represented by the longest chain, which has the greatest proof-of-work effort invested
in it.
If a majority of CPU power is controlled by honest nodes, the honest chain will grow the
fastest and outpace any competing chains.
To modify a past block, an attacker would have to
redo the proof-of-work of the block and all blocks after it and then catch up with and surpass the
work of the honest nodes.
We will show later that the probability of a slower attacker catching up
diminishes exponentially as subsequent blocks are added.

To compensate for increasing hardware speed and varying interest in running nodes over time,
the proof-of-work difficulty is determined by a moving average targeting an average number of
blocks per hour.
If they're generated too fast, the difficulty increases.

\section{Network}\label{sec:network}
The steps to run the network are as follows:
    \begin{enumerate}[label={\arabic*)}] % https://tex.stackexchange.com/questions/42905/enumerated-list-with-square-brackets
        \item New transactions are broadcast to all nodes.
        \item Each node collects new transactions into a block.
        \item Each node works on finding a difficult proof-of-work for its block.
        \item When a node finds a proof-of-work, it broadcasts the block to all nodes.
        \item Nodes accept the block only if all transactions in it are valid and not already spent.
        \item Nodes express their acceptance of the block by working on creating the next block in the
    chain, using the hash of the accepted block as the previous hash.
    \end{enumerate}

Nodes always consider the longest chain to be the correct one and will keep working on
extending it.
If two nodes broadcast different versions of the next block simultaneously, some
nodes may receive one or the other first.
In that case, they work on the first one they received,
but save the other branch in case it becomes longer.
The tie will be broken when the next proof-
of-work is found and one branch becomes longer; the nodes that were working on the other
branch will then switch to the longer on.

New transaction broadcasts do not necessarily need to reach all nodes.
As long as they reach
many nodes, they will get into a block before long.
Block broadcasts are also tolerant of dropped
messages.
If a node does not receive a block, it will request it when it receives the next block and
realizes it missed one.

\section{Incentive}\label{sec:incentive}
By convention, the first transaction in a block is a special transaction that starts a new coin owned
by the creator of the block.
This adds an incentive for nodes to support the network, and provides
a way to initially distribute coins into circulation, since there is no central authority to issue them.
The steady addition of a constant of amount of new coins is analogous to gold miners expending
resources to add gold to circulation.
In our case, it is CPU time and electricity that is expended.

The incentive can also be funded with transaction fees.
If the output value of a transaction is
less than its input value, the difference is a transaction fee that is added to the incentive value of
the block containing the transaction.
Once a predetermined number of coins have entered
circulation, the incentive can transition entirely to transaction fees and be completely inflation
free.

The incentive may help encourage nodes to stay honest.
If a greedy attacker is able to
assemble more CPU power than all the honest nodes, he would have to choose between using it
to defraud people by stealing back his payments, or using it to generate new coins.
He ought to find it more profitable to play by the rules, such rules that favour him with more new coins than
everyone else combined, than to undermine the system and the validity of his own wealth.

\section{Reclaiming Disk Space}\label{sec:reclaiming-disk-space}
Once the latest transaction in a coin is buried under enough blocks, the spent transactions before
it can be discarded to save disk space.
To facilitate this without breaking the block's hash,
transactions are hashed in a Merkle Tree [7][2][5], with only the root included in the block's hash.
Old blocks can then be compacted by stubbing off branches of the tree.
The interior hashes do not need to be stored.

\usetikzlibrary{
arrows.meta,% [>=Triangle]
fit,        % [fit=...]
positioning % [right=of ...]
}

\begin{tikzpicture}[>=Triangle]
\tikzset{box/.style={draw, minimum size=1em, text centered},
    container/.style={draw, inner sep=18pt}
}
% Prevent childs from overlepping.  https://tex.stackexchange.com/a/60579/264984
\tikzstyle{level 1}=[<-, sibling distance=36mm]
\tikzstyle{level 2}=[<-, sibling distance=15mm]

\begin{scope}[xshift=0cm]
    \node (RootHash) [draw]{Root Hash}
    child {node [draw, dashed]{Hash01}
        child {node (Hash0) [draw, dashed]{Hash0}}
        child {node (Hash1) [draw, dashed]{Hash1}}
    }
    child {node [draw, dashed]{Hash23}
        child {node (Hash2) [draw, dashed]{Hash2}}
        child {node (Hash3) [draw, dashed]{Hash3}}
    };

    \node (Tx0) [below=of Hash0] [draw, inner sep=9] {Tx0};
    \node (Tx1) [below=of Hash1] [draw, inner sep=9] {Tx1};
    \node (Tx2) [below=of Hash2] [draw, inner sep=9] {Tx2};
    \node (Tx3) [below=of Hash3] [draw, inner sep=9] {Tx3};

    \draw [<-] (Hash0.south) -- (Tx0.north);
    \draw [<-] (Hash1.south) -- (Tx1.north);
    \draw [<-] (Hash2.south) -- (Tx2.north);
    \draw [<-] (Hash3.south) -- (Tx3.north);

    \node (PrevHash) [above=0.3cm of RootHash] [draw] {Prev Hash};
    \node (Nonce) [right=of PrevHash] [draw] {Nonce};
    \node (BlockHeaderBefore) [fit=(PrevHash)(Nonce)(RootHash)]
    [container, label={[shift={(0ex,-4ex)}]north:Block Header(BlockHash)}]{};
    \node (TreeBefore)[container,fit=(BlockHeaderBefore)(Tx0)(Tx3)][label={[shift={(8ex,-4ex)}]north west:Block}]{};
    \node [below=0.3cm of TreeBefore] {Transactions Hashed in a Merkle Tree};
\end{scope}

\begin{scope}[xshift=8cm]
    \node (RootHash) [draw]{Root Hash}
    child {node (Hash01)[draw]{Hash01}}
    child {node [draw, dashed]{Hash23}
    child {node (Hash2) [draw]{Hash2}}
    child {node (Hash3) [draw, dashed]{Hash3}}
    };

    \node (TTx3) [below=of Hash3] [draw, inner sep=9] {Tx3};

    \draw [<-] (Hash3.south) -- (TTx3.north);

    \node (PrevHash) [above=0.3cm of RootHash] [draw] {Prev Hash};
    \node (Nonce) [right=of PrevHash] [draw] {Nonce};
    \node (BlockHeaderAfter) [fit=(PrevHash)(Nonce)(RootHash)]
    [container, label={[shift={(0ex,-4ex)}]north:Block Header(BlockHash)}]{};
    \node (TreeAfter)[container,fit=(BlockHeaderAfter)(Hash01)(TTx3)][label={[shift={(8ex,-4ex)}]north west:Block}]{};
    \node [below=0.3cm of TreeAfter] {After Pruning Tx0-2 from the Bloc};
\end{scope}

\end{tikzpicture}


A block header with no transactions would be about 80 bytes.
If we suppose blocks are generated every 10 minutes, 80 bytes * 6 * 24 * 365 = 4.2MB per year.
With computer systems typically selling with 2GB of RAM as of 2008, and Moore's Law predicting current growth of
1.2GB per year, storage should not be a problem even if the block headers must be kept in memory.

\end{document}